\documentclass[12pt]{ctexart}
\usepackage{amsmath}
\usepackage{amssymb}
\usepackage{amsthm}
\usepackage{booktabs}
\usepackage{pifont}
\usepackage{enumitem}
\usepackage{esint}
\usepackage{fancyhdr}
\usepackage{float}
\usepackage{fontspec}
\usepackage{geometry}
\usepackage{hyperref}
\usepackage{import}
\usepackage{indentfirst}
\usepackage{lastpage}
\usepackage{nicematrix}
\usepackage{physics}
\usepackage{setspace}
\usepackage{subcaption}
\usepackage{tikz}
\usetikzlibrary{shapes}
\usepackage{titlesec}
\usepackage{titletoc}
\usepackage{wrapfig}
\usepackage{xcolor}
\usepackage{xparse}
\usepackage{xstring}
\usepackage{zhnumber}
\geometry{a4paper,top=2cm,bottom=3.2cm,inner=2.5cm,outer=3.2cm}
\setmonofont{Fira Code}
\setCJKmainfont{Noto Serif CJK SC}
\setCJKsansfont{Noto Sans CJK SC}
\setCJKmonofont{Fira Code}
\DeclareMathOperator{\ee}{\mathrm{e}}
\DeclareMathOperator{\ii}{\mathrm{i}}
\DeclareMathOperator{\jj}{\mathrm{jj}}
\renewcommand{\thesection}{\zhnum{section}}
\titleformat{\section}{\normalsize}{\bfseries\thesection、}{0.25em}{}
\titleformat{\subsection}{\normalsize}{\bfseries\arabic{subsection}.}{1em}{}
\titleformat{\subsubsection}{\normalsize}{\bfseries\hspace{2em}(\arabic{subsubsection})}{.5em}{}
\allowdisplaybreaks[4]
\setstretch{1.5}
\setlength{\parskip}{0.4em}
\everymath{\displaystyle}
\NewDocumentCommand{\options}{O{4}mmmm}{
	\IfEqCase{#1}{
		{4}{
			\begin{table}[H]
				\centering
				\begin{NiceTabular}[width=\linewidth]{@{}!{\qquad}*{4}{X[1,l]}}
					A. #2 & B. #3 & C. #4 & D. #5
				\end{NiceTabular}
			\end{table}
		}
		{2}{
			\begin{table}[H]
				\centering
				\begin{NiceTabular}[width=\linewidth]{@{}!{\qquad}*{2}{X[1,l]}}
					A. #2 & B. #3 \\
					C. #4 & D. #5
				\end{NiceTabular}
			\end{table}
		}
		{1}{
			\begin{table}[H]
				\centering
				\begin{NiceTabular}[width=\linewidth]{@{}!{\qquad}X[1,l]}
					A. #2 \\
					B. #3 \\
					C. #4 \\
					D. #5
				\end{NiceTabular}
			\end{table}
		}
	}
}
\NewDocumentCommand{\blank}{}{\underline{\qquad}}

\title{北京邮电大学2023-2024学年第二学期\\《高等数学A(下)》期末考试试题(A卷)\\参考答案}
\author{制作:\textit{cppHusky}}
\date{}
\begin{document}
\pagestyle{fancy}
\fancyhf{}
\fancyhead[L]{\footnotesize\sffamily 23-24第二学期\ 高等数学A(下)期末A卷答案}
\fancyhead[R]{\footnotesize\sffamily 制作: \textit{cppHusky}}
\fancyfoot[C]{\footnotesize\sffamily 第\thepage 页,共\pageref*{LastPage} 页}
\fancyfoot[R]{\footnotesize \href{https://byrdocs.org/}{byrdocs.org}}
\maketitle
\thispagestyle{fancy}
\section{填空题 (每小题3分, 共30分)}
\subsection{\(\frac{2}{3}\).}
\subsection{发散.}
\subsection{\(p\ge1\).}
\subsection{\(\frac{3}{2}\).}
\subsection{\(4\ee^{4}\).}
\subsection{\(\frac{x-2}{-3}=\frac{y+1}{-1}=\frac{z-1}{5}\).}
\subsection{\(0\).}
\subsection{\(2\pi\).}
\subsection{\(\frac{3}{4}\).}
\subsection{\(-1-\ee\).}
\pagebreak
\section{(10分)}
全微分为: \(\dd{z}=\qty[f_{1}+2f_{2}+\varphi'(g)g_{1}]\dd{x}+\qty[-f_{1}+f_{2}+\varphi'(g)g_{2}]\dd{y}\).
\begin{align*}
	\pdv{z}{x}{y}=-f_{11}+f_{12}-2f_{21}+2f_{22}+\varphi''(g)g_{1}^{2}+\varphi'(g)g_{12}.
\end{align*}
又因为\(f\)的二阶偏导数连续,所以\(f_{12}=f_{21}\), 因此
\begin{align*}
	\pdv{z}{x}{y}=-f_{11}-f_{12}+2f_{22}+\varphi''(g)g_{1}^{2}+\varphi'(g)g_{12}.
\end{align*}\par
\section{(10分)}
收敛半径: \(R=\lim_{n\rightarrow\infty}\abs{\frac{(-1)^{n-1}(n+1)n}{(-1)^n(n+2)(n+1)}}=1\).\par
当\(x=\pm1\)时, \(\lim_{n\rightarrow\infty}(-1)^{n-1}(n+1)n=\infty\), 此时级数不收敛.\par
综上所述, 级数\(\sum_{n=1}^{\infty}(-1)^{n-1}(n+1)nx^{n}\)的收敛域为\((-1,1)\).\par
记和函数为\(S(x)\), 则\[\frac{S(x)}{x}=\sum_{n=1}^{\infty}(-1)^{n-1}(n+1)nx^{n-1}=\dv[2]{x}\sum_{n=1}^{\infty}(-x)^{n+1}=\dv[2]{x}\frac{1}{1+x}.\]于是\(S(x)=\frac{2x}{(x+1)^{3}}\).\par
代入\(x=-\frac{1}{3}\)可以得到\(S\qty(-\frac{1}{3})=\sum_{n=1}^{\infty}\frac{-(n+1)n}{3^{n}}=-\frac{9}{4}\), 所以\(\sum_{n=1}^{\infty}\frac{(n+1)n}{3^{n}}=\frac{9}{4}\).\par
\section{(10分)}
先将积分区域分为两部分:
\begin{align*}
	\text{I}:{}&1\le x^{2}+y^{2}\le3, x+y\ge0, x\ge y;\\
	\text{II}:{}&1\le x^{2}+y^{2}\le3, x+y\ge0, x\le y.
\end{align*}
因为\(x^{3}\ln(y+\sqrt{y^{2}+1})\)是关于\(x\)和\(y\)的奇函数, 所以对于I区域 (关于\(x\)轴对称) 和II区域 (关于\(y\)轴对称) 各有\[
	\iint x^{3}\ln(y+\sqrt{y^{2}+1})\dd{x}\dd{y}=0.
\]
而这个积分的第一项只需使用极坐标形式求解即可:
\begin{align*}
	\iint\frac{x}{1+x^{2}+y^{2}}\dd{x}\dd{y}={}&\iint\frac{\rho^{2}\cos{\theta}}{1+\rho^{2}}\dd{\rho}\dd{\theta}\\
	={}&\int_{-\frac{\pi}{4}}^{\frac{3\pi}{4}}\dd{\theta}\int_{1}^{3}\frac{\rho^{2}}{1+\rho^{2}}\dd{\rho}\\
	={}&2-\arctan(3)+\frac{\pi}{4}.
\end{align*}\par
\section{(10分)}
换序即可.
\begin{align*}
	{}&\int_{0}^{1}\dd{y}\int_{0}^{1-y}\dd{z}\int_{0}^{1-y-z}(1-x)\ee^{-(1-x-z)^{2}}\dd{x}\\
	={}&\int_{0}^{1}(1-x)\dd{x}\int_{0}^{1-x}\ee^{-(1-x-z)^{2}}\dd{z}\int_{0}^{1-x-z}\dd{y}\\
	={}&\int_{0}^{1}(1-x)\dd{x}\int_{0}^{1-x}(1-x-z)\ee^{-(1-x-z)^{2}}\dd{z}\\
	={}&\frac{1}{2}\int_{0}^{1}(1-x)\qty[1-\ee^{-(1-x)^{2}}]\dd{x}\\
	={}&\frac{1}{4\ee}.
\end{align*}\par
\section{(10分)}
由格林公式可知,\[
	\oint_{L}(1-x^{2}y)\dd{x}+(1+xy^{2})\dd{y}=\iint_{S}(y^{2}+x^{2})\dd{x}\dd{y},
\]其中\(S: x^{2}+y^{2}\le2x\).\par
再用极坐标形式求解: \[
	\iint_{S}(y^{2}+x^{2})\dd{x}\dd{y}=\iint_{S'}\rho^{3}\dd{\rho}\dd{\theta}=\int_{-\frac{\pi}{2}}^{\frac{\pi}{2}}\int_{0}^{2\cos{\theta}}\rho^{3}\dd{\rho}\dd{\theta}=\frac{3\pi}{2}.
\]
\section{(10分)}
根据对称性,\[
	\iint_{\Sigma}(x+x^{3})\dd{x}\dd{y}+(y+y^{3})\dd{z}\dd{x}=0.
\]\par
于是\(I=-2\iint_{S}\sqrt{x^{2}+y^{2}}\dd{x}\dd{y}\), 再用极坐标求解得\[
	I=-2\int_{0}^{2\pi}\dd{\theta}\int_{0}^{1}\rho^{2}\dd{\rho}=-\frac{4\pi}{3}.
\]
\section{(10分) (本题答案由 Proleta 给出)}
\(\grad{f}=(2+y,3+x)\), 所以\(\norm{\grad{f}}=\sqrt{(3+x)^{2}+(2+y)^{2}}\). 要在限制条件\(x^{2}+y^{2}=52\)下求出\(\norm{\grad{f}}\)的最大值, 应使用拉格朗日乘数法.\par
考虑
\begin{align*}
	\begin{cases}
		g(x,y)=(3+x)^{2}+(2+y)^{2},\\
		L(x,y,\lambda)=(3+x)^{2}+(2+y)^{2}+\lambda(x^{2}+y^{2}-52),
	\end{cases}
\end{align*}
有
\begin{align*}
	\begin{cases}
		L_{x}=2(x+3)+2\lambda x=0,\\
		L_{y}=2(y+2)+2\lambda y=0,\\
		L_{\lambda}=x^{2}+y^{2}-52=0,
	\end{cases}
\end{align*}
解出
\begin{align*}
	\begin{cases}
		x=-6,\\
		y=-4,\\
		\lambda=-\frac{1}{2}\\
	\end{cases}
	\qq{或}
	\begin{cases}
		x=6,\\
		y=4,\\
		\lambda=-\frac{3}{2}.\\
	\end{cases}
\end{align*}\par
经过验证, \(\norm{\grad{f(-6,-4)}}=\sqrt{13},\;\norm{\grad{f(6,4)}}=3\sqrt{13}\). 所以\(f(x,y)\)在圆周\(C\)上方向导数的最大值是\(3\sqrt{13}\).\par
\end{document}
